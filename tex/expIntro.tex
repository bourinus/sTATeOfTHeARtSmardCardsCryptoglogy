\subsubsection*{Warning !}
Note that of the following algorithms might be implemented indifferently
in left-to-right or for right-to-left, or also in atomic or non atomic
version, and so on and so on... A lot of combination are possible. Warning
\subsubsection{Convention and Names !}

The famous routine \textbf{"Square \& Multiply"} was named this way for a descriptive purpose: when we are using it we are effectively squaring and multiplying.

On the other hand it's mathematical name is \textbf{"dichotomic exponentiation"}
insisting on the fact that to achieve the exponentiation, is proceeded to recursive
calls to a sub-routine and that for each call there are two possibility 
multiplication.

The name \textbf{"binary method"} has been given with the same spirit.
Starting from now we adopt this kind of naming for every algorithm relative to
exponentiation. In this section will be viewed other algorithm scanning not
one bit but several bits of the exponent at each recursion. To have a clear
naming we extend the previous convention to this other class of algorithms
with the following convention:

\begin{center}
\textbf{2-ary method:} each recursion
$2^1$ possibilities and bit scanned $1$ by $1$. \\
\textbf{$ \mathbf{2^k}$-ary method:} each recursion
$2^k$ possibilities and bit scanned $k$ by $k$.
\end{center}
\vspace{3mm}
Some might object that could be said that a 3-arry method is scanning bits 3 by 3.\\
\underline{Example:} What about $ x ^ {27} $ without multiplying?\\
If you have an efficient algorithm to compute third power and
you build on it an exponentiation algo with at each recursion
a possible multiplication by $x^0$, $x^1$, $x^3$.
This algorithm works in a representation of 27 in base 3, and can't be named
with the other convention.
It's named trichotomic exponentiation, ternary method or cube and multiply.
\vspace{3mm}