\item Boreale - \lokiquote{fdtc-2006-boreale}

\textbf{Paper's sum-up:}\\
Present a FI attacks breaking straightforward RSA in RtoL version using the Jacobi Symbol, using
a practical fault model: that external perturbation, or glitch, may cause a single modular 
multiplication to produce a truly random result. Two attacks are presented, the second one having relaxed condition.

\underline{Type of attack:}
known $\&$ reproducible plaintext attack .

\underline{Targetted:} RtoL $S\&M$ only.

\underline{Counter measures:}\\
Unefficient: Blind masking of the message: $m$ replaced by $r^e \times m \mod N$.\\
Efficient: verify the signature by checking that $S^e = m \mod N$, exponnent masking,
various delays, modulus blinding-?-.

\underline{Fault model:}
Practical: change the result of a certain multiplication for a random one.

\underline{Mathematical background:}
Jacobi symbol generalizes Legendre's ones, which value 
$\left( \frac{a}{p} \right)$, for $p$ prime, means:
\begin{center}
$
\left( \frac{a}{p} \right) = \left\{
    \begin{array}{ll}
      \;\; 1 & \textsf{if} \;\;\exists \; x \neq 0 \in \mathbb{Z}/{p\mathbb{Z}}  / a = x^2 \mod p\\
      -1     & \textsf{if} \not{\exists} \; x \neq 0 \in \mathbb{Z}/{p\mathbb{Z}}  / a = x^2 \mod p\\
       \;\;0 & \textsf{if} \;\;a = 0 \mod p\\
    \end{array}
\right.
$
\end{center}


\href{http://en.wikipedia.org/wiki/Jacobi_symbol}{ A link: Wiki about Jacobi \& legendre symbols: }

\textbf{Summing up \textit{Attack}} LtoR $S\&M$
		
Assumptions:\\
1- Each modular multiplication/squaring operation takes a constant time, 
say $\delta$ clock cycles, and $\delta$ is a constant known to the attacker.\\
2- Time taken by control-flow instructions is ignored, we view the
algorithm as a sequence of modular multiplications. 
Each phase $i$ takes either $\delta$ or $2\delta$.\\
3- A glitch applied onto the device during the execution of a modular multiplication
will result in a random value $r\in \mathbb{Z}/{2^s\mathbb{Z}}$ to be written in 
the involved register in place of the multiplication's correct result.\\
4- For message m it is assumed that $\left( \frac{m}{N} \right)=1$, 
if equal to $-1$ some equation shall be slighlty modified, the case where 
$\left( \frac{m}{N} \right)=0$ in unlikely : it implies $m=p  \textsf{ or } q$

The authors begin to define $T_i$ the moment when happend the $i^{th}$ operation 
while an encryption is performed. As the attacker already knows $d_0$ 
the first fault is done around  $t=T_1$, more precisely, $T_1 > t > T_1 -\delta$,
this will provok a fault in the squarring of the phase $i-1$.

The obtained signature can be written the following way, using the classical notation 
$c_i = m^{2^i} \mod n$
\begin{center}
$
S^{'} = 
c_0^{d_0}   c_1^{d_1}     \; \hdots \;         c_{i-1}^{d_{i-1}}
(r)^{d_i} (r^2)^{d_{i+1}} \; \hdots \; (r^{2^{l-i-1}})^{d_{l-1}}
 \mod n
$
\end{center}
Taking in account hypothesis 4, then it is clear that, except $ (r)^{d_i} $, 
each divisor of $S^{'}$ has Jacobi Symbol different from -1.
Therefore $\left( \frac{S^{'}}{N} \right)= -1$ implies
$\left( \frac{r^{d_i}}{N} \right)= -1$ and then $d_i = 1$. 
On the other hand to obtain $\left( \frac{S^{'}}{N} \right) \neq -1$ 
suggest that the more probable is that $d_i =0$. 

Authors finishes this part with an evaluation of the
probability $\{ d_i=1 \| \left( \frac{S^{'}}{N} \right) \neq -1 \}$.

If the moment of the $i^{th}$ operation is difficult to estimate,
the attack is run severasl time $ \approx 50$.

\underline{Software simulation:}\\
On a 768-RSA, 5000 faults are enough to recover, in 30 minutes,
the whole key in 70\% of the cases.





	
	
