\newpage
\section{Asymmetrical implementation in smart cards}
\label{Asymmetrical_implementation_in_smart_cards}

\textit{Notation:}
In the following part of the text:
\begin{itemize}
	\item $( \mathbb{G},\; \times )$ is a commutative group
	\item $x$, $y$ and $z$ three elements belonging to this group
	\item $n$ an positive integer
	\item $ln_b(x)$ is the Neperian logarithm in base $b$ of $x$ 
		\textit{i.e.} $ln_b(x)= ln(x)/ln(b)$ 
\end{itemize}
Without an explicit mention those algorithms work for every commutative group
\footnote{On of the first property of commutative, or Abelian groups, is to give a sense to Binomial theorem and its generalization (multinomial theorem)} $ \mathbb{G} $.
Note that is was not always possible to say 'the improvement of this algorithm
lies in the multiplication only', so be careful. By example the Infineon technologies ZDN algorithm is an algorithm improving at the same time, multiplication and reduction.
\vspace{5mm}



		