\newpage
\subsection{what exponentiation is about}
\label{RSA_what_exponentiation_is_about}

\cite{unibordeau-2006-bergson}\\
\cite{preprint-2006-bernstein}\\



The hereafter section is a little bit more theoretical	but
necessary to understand what exponentiation is about.
After each example, we saw the key role of intermediate exponentiations to obtain 
the desired final result, study intermediate computation is the key for very fast 
exponentiation.

In fact this problem is a very good example of theoretical mathematics that have 
a very concrete application. The concrete application in which we are interested
is answer the to the following question: 

what is the minimal number of multiplications 
necessary to perform a given exponentiation ?

\subsubsection*{Addition chain}	
		
Let's formalize this: due to the propriety of the exponential function this
problem can be described as a problem of addition, and conduct to the following definition:

\textit{Definition:}
	A finite sequence of positive integers $1=a_0$, $a_1$, $a_r=n$ is called an 
	addition chain for the number $n$ iff for each element $a_i$, but the first $a_0$, 
	there exist two elements in the list $a_j$ and $a_k$ such that:
	\begin{center}
	$a_i  = a_j+a_k$ with $k \leq j \leq i-1$
	\end{center}
	For example $\{\;1,\; 2,\; 4,\; 5,\; 8,\; 10,\; 13\}$, length 7.
	
\textit{Definition:}
	The shortest addition chain for n is an addition chain for n 
	with the smallest possible number of elements.
	The length of this shortest addition chain for $n$ is noted $l(n)$. \\ 
	For example $\{\;1,\; 2,\; 3,\; 6,\; 12,\; 13\}$, length $6=l(13)$.
%		\begin{algorithm}[h]
%			\caption{Exponentiation in term of Addition chain}
%			\begin{algorithmic}
%			\REQUIRE{$x,$ an addition chain $,\{ n_{1} , ...,n_l\},$ computing $d$}
%			\ENSURE{$ y   =  x^d \mod n$} 
%			\STATE $y \leftarrow 1$		
%				\FOR{$i = 1 \to l$} 
%				 	\STATE $(s,r)$ as $n_i = n_s + n_r$
%					\STATE $y \leftarrow x^{n_s} \times x^{n_r} $	
%				\ENDFOR
%			\RETURN{$ y $}
%			\end{algorithmic}
%		\end{algorithm}\\
\begin{algorithm}[h]
	\KwIn{$x,$ an addition chain $,\{ n_{1} , ...,n_l\},$ computing $d$}
	\KwOut{ $ y   =  x^d \mod n$ }
	 $y \leftarrow 1$ \;
	\For{$i  = 1$ \textbf{to} $l$}
	{
	$(s,r)$ as $n_i = n_s + n_r$
	$y \leftarrow x^{n_s} \times x^{n_r} $				
	}									 
	\Return{$ y $}\;
	\caption{Exponentiation in term of Addition chain}
\end{algorithm}
		
\textit{Definition:}
		A Brauer chain is an addition chain in which every member after the 
		first is the sum of the immediately preceeding element and a 
		previous element (possibly the same element). More formally: 		 
		\begin{center}
		$a_i  = a_{i-1}+a_k$ with $k \leq i-1 $
		\end{center}	
		For example $\{\;1,\; 2,\; 3,\; 6,\; 7,\; 13\}$ is one,
		$\{\;1,\; 2,\; 4,\; 5,\; 8,\; 13\}$ is not.
			 
\underline{Example:} Shortest addition chain for $n=23_{10}=10111_2$ \\ 
		Saved value: value: $x^{4},\;x^{5}$.\\			
		\begin{tabularx}{\linewidth}{ p{16cm} p{2cm} }
			$y:=y \times x$ 		& $(y=x^{1})$\\
			$y:=y^2$ 				& $(y=x^{4})$\\
			$y:=y \times x$ 		& $(y=x^{5})$\\
			$y:=x^{4} \times x^{5}$ & $(y=x^{9})$\\
			$y:=y^2$ 				& $(y=x^{18})$\\
			$y:=y \times x^{5}$     & $(y=x^{23})$				
		\end{tabularx}
		Remark that the following intermediate exponentiation were computed:			
		$ x^{1},\; x^{2},\; x^{4},\; x^{5},\; x^{9},\; x^{18},\; x^{23} $,
		length 7. 
		\vspace{3mm}
			
\subsubsection*{Addition/Subtraction chain}	

\subsubsection*{Practical consideration}

to find the shortest addition chain is 'NP difficult', and interesting algorithm
shall be hardware compatible (no smart car runs a 5-ary method with 5 choices of
multiplication at each iteration of the algorithm or at least its seems).

A same addition chain can be computed with different number of step depending 
on the algorithm (S\&M \textit{vs} Atomic S\&M), changing the speed but impacting also the side channel leakage.


	
	 also exists other techniques fastening the computation of a
	 product of several exponentiation at the same time Strauss -1964-,Yao and Pippenger -1976-

