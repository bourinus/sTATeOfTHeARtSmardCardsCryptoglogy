\newpage
\subsection{Reduction}
\label{RSA_reduction}

Problem: $ T $ a number to reduce modulo $ N $, respectively $2n$ and $n$ bit long.\\
Aim: to deal efficiently with reduction modulo $N$, computation in $\mathbb{Z}/{N \mathbb{Z}}$.\\\\
\underline{References:}\\
\cite{crypto-1993-bosselaers}\\
\cite{arith-2007-hasenplaugh}\\
\cite{eprint-2014-zhengjun}

Duality between Multiplication and Modular Reduction: \cite{eprint-2005-fisher}
\begin{itemize}
	\item 
		\begin{tabularx}{\linewidth}{ p{14.5cm} p{3cm} }
		Euclidean method method -
		\textit{2000BC}- & $0\%$ 
	 	\end{tabularx}
	 		A complete Euclidean division is achieved:\\
	 		$ T = q N + r $ with $ 0 \leq r < N$\\\\
			\underline{Example:}\\
			Let's to be computed $x.y \mod N$  with $N = 119$, and $x = 63$, $y = 57$:\\
			$63.57 \mod 119 = 3591 \mod 119 = 119.30+21 \mod 119 = 21 \mod 119$\\
			But this requires a costly division which process is hidden...
		 		 
	\item
		\begin{tabularx}{\linewidth}{ p{14.5cm} p{3cm} }
		Knuth's "scholar" method - \textit{1969}- & $0\%$	  
		\end{tabularx}
		  	Improve the previous algorithm in the sense that only a 
		  	partial Euclidean division is performed.	  	
			
	\item 
		\begin{tabularx}{\linewidth}{ p{14.5cm} p{3cm} }
		Montgomery's reduction - \cite{ieeetc-2005-montgomery} - & $60\%$ \\
		\end{tabularx}
			Montgomery reduction algorithm is a pillar of modern modular calculus.
			It exists in many versions, with computation by bloc or globally
			with interleaved multiplication or not, and many more.\\

			Computation by isomorphism: computation are not performed in the original group 
			$\mathbb{Z}/{N \mathbb{Z}}$ but rather in another 
			representation of this ring.\\
			for $R$ co-prime with $N$, group isomorphism
			for $R$ co-prime with $N$ :
			$
			\phi\left\{
			    \begin{array}{c}
			     \mathbb{Z}/{N \mathbb{Z}} \longrightarrow {\mathbb{Z}/{N \mathbb{Z}}}^{'}  \\
			      \;\;\;\;x \longmapsto x.R
			    \end{array}
			\right.
			$
		    the operation corresponding
		    	    \footnote{
		    	Since $\phi(a).\phi(b) = a.b.R^2$, and  $\phi^{-1}(a.b.R^2)=a.b.R$
		    	}to the multiplication in $\mathbb{Z}/{n \mathbb{Z}}$
		    is $M_R (a,b)\mapsto a.b/R$
	
		    	   		  		   
			\begin{algorithm}
				\KwIn{
					$ T \in \mathbb{Z}/{R.N \mathbb{Z}}$ to reduce modulo $ N $ \\
					$ R, N \in \mathbb{Z}/{N \mathbb{Z}}$ such that $ pgcd(R,N) =1$\\
					$ N^{'} \in \mathbb{Z}/{R \mathbb{Z}}$ such that $R.R^{'}-N.N^{'}=1$
				}
				\KwOut{$ T.R^{-1} \mod  N $ }
				$m \leftarrow	 T.N^{'} \mod R$ \;
				$t \leftarrow	 (T+m.N).R^{-1}$ \;
				\If {$t \geq n$}{
					$ t \leftarrow T \mod  N $\;
					} 
				\Return{$ t $}\;
				\caption{Montgomery's reduction algorithm $redct$ function}
			\end{algorithm}				
			we have  an algorithm:
			\begin{itemize}
				\item achieving a division by $R$, that can be chosen under condition
				\item reducing its input modulo $N$
				\item without divisions by $N$, if $T \leq R.N$
			\end{itemize}

		\textbf{Proof:}
			It is clear that $ T = T + m.N \mod N $, with $R.R^{'}-N.N^{'}=1$ and 
			$ m = T.N^{'} \mod R$ we have: $  T + m.N = T- (T \mod R)$ then 
			$  T + m.N \leq RN + RN$ so that $ 0 \leq t \leq 2N $.

			Function  $redct$is used to multiply in Montgomery domain but also 
			to as 'transfer':\\
			Thanks to: $ redct(x.R^2) = x.R \mod N = \phi(x) $ and $redct(x) = x.R^{-1} \mod N = \phi^{-1}(x) $.

			In practice we can choose $R$ regarding the modulus pre-compute $R^2 \mod N$, $R^{-1} \mod N$\\
 		  
		\underline{Example:}
			Let's to be computed $x.y \mod N$  with $N = 119$, and $x = 63$, $y = 57$:\\
			Define isomorphism :
			choose a hardware compatible $R$ with the condition $R > N$,  $R =2^7 = 128$\\ 
			Precomputed values : $R^2 \mod N = 81$, $R^{-1} \mod  N = 53$\\
			Transfer data into Montgomery Domain: 
			\begin{center}
				$ \phi(x) = x.R \mod N  = redct(x.R^2) = redct(63.81) = 91 $\\
				$ \phi(y) = y.R \mod N  = redct(y.R^2) = redct(57.81) = 37$\\
			\end{center}
			Multiplication in Montgomery domain: 
			\begin{center}
				$ M_R(\phi(x),\phi(y)) = redct(91.37.53) \mod 119 = 70$
			\end{center}
			Return the result from the Montgomery domain: 
			\begin{center}
				$x.y \mod N = redct(70) = 21$	
			\end{center}
			
			All the division was by $R$, if $R$ is a power of two for the hardware this only a shift. 
			Secondly $R^{2}$ and $R^{-1}$ are precomputed values.	  
					   

	\item \begin{tabularx}{\linewidth}{ p{14.5cm} p{3cm} }
		Barrett's reduction -\textit{\cite{crypto-1986-barrett}} & Widely used 
		\end{tabularx}
		Involves one pre-computation: 
		$\mu = \lfloor \frac{2^{2n}}{N} \rfloor$.\\
		The reduction take the form:
		\begin{center}
			$R = T - \lfloor \lfloor \frac{T}{2^{n}} \rfloor \frac{\mu}{2^{n}} \rfloor N$\\
		\end{center}		

		Requires: two +1 n-bit multiplications and on subtraction.

		\underline{Example:}
			$x.y \mod N$  with $N = 119$, and $x = 63$, $y = 57$\\
			Precomputed values :
			$\mu  = \lfloor 2^{16}/119 \rfloor  = 550$\\
			
			The reduction take the form\\
			$R = 3591 - \lfloor \lfloor \frac{3591}{2^{8}} \rfloor \frac{550}{2^{8}} \rfloor 119$\\
			$R = 3591 - \lfloor 14 \times 2.148 \rfloor 119$\\
			$R = 3591 - 30.119$\\
			$R = 3591 - 3570$\\
			$R = 21 $ 


	\item \begin{tabularx}{\linewidth}{ p{14.5cm} p{3cm} }
		Iterative folding -\textit{\cite{arith-2007-hasenplaugh}} & $0\%$ 
		\end{tabularx}
		Iterates a divide and conquer approach to have smaller multiplication, in Barrett's algorithm. This involve more pre-computations: 
		$\forall i, 1 \leq i \leq F, M^{(i)} = 2^{(1+2^{-i})n}$ \\
		and for the final Barrett step :
		$ \mu = \lfloor \frac{2^{2n}}{M}  2^{2^{-F}n}\rfloor$\\

		The reduction take the form\\
		$N^{(0)} =N$\\
		$N^{(i)} =N^{(i-1)} \mod M^{(i)} + \lfloor \frac{N^{(i-1)}}{M^{(i)}}  \rfloor M^{(i)}\forall i, 1 \leq i \leq F$\\
		$	R = N^{(F)} - \lfloor \lfloor \frac{N^{(F)}}{2^{(1+2^{-F})n}} \rfloor \frac{\mu}{2^{2^{-F}n}} \rfloor M$\\

		Optimal for $F=2$

		\underline{Example:}
			Let's to be computed $x.y \mod N$  with $N = 119$, and $x = 63$, $y = 57$:\\
			Precomputed values : $N$ is 8 digits, we set $F=2$:\\
		$
		M^{(1)} = 2^{(1+2^{-1})8} = 2^{12} = 4096 \\
		M^{(2)} = 2^{(1+2^{-2})8} = 2^{10} = 1024$ \\
		And for the final Barrett step :\\
		$ \mu = \lfloor \frac{2^{16}}{119} 2^{2^{-2}8} \rfloor = \lfloor \frac{2^{16}}{119} 4 \rfloor = 2202$\\

		The reduction take the form\\
		$N^{(0)}= 3591$\\
		$N^{(1)}= 3591 \mod 2^{12} 	+ \lfloor \frac{3591}{2^{12}}  \rfloor M^{(1)}
				= 3591 \mod 2^{12} 				+ \lfloor \frac{3591}{2^{12}}  \rfloor 4096 = 3591$\\
		$N^{(2)}= N^{(1)} \mod 2^{(1+2^{-2})8} 	+ \lfloor \frac{3591}{2^{10}}  \rfloor M^{(2)}
				= 3591 \mod 2^{10} 				+ \lfloor \frac{3591}{2^{10}}  \rfloor 1024 = 519 + 3072 =3591 $\\		
		$	R   = N^{(2)} - \lfloor \lfloor \frac{N^{(2)}}{2^{(1+2^{-2})n}} \rfloor \frac{\mu}{2^{2^{-2}n}} \rfloor M$\\
		$	R   = 3591 - \lfloor \lfloor \frac{3591}{2^{10}} \rfloor \frac{\mu}{2^{2}} \rfloor 119$\\
		$	R   = 3591 - 3 550.5* 119$\\


			$\mu  = \lfloor 2^{16}/119 \rfloor  = 550$\\
			
			The reduction take the form\\
			$R = 3591 - \lfloor \lfloor \frac{3591}{2^{8}} \rfloor \frac{550}{2^{8}} \rfloor 119$\\
			$R = 3591 - \lfloor 14 \times 2.148 \rfloor 119$\\
			$R = 3591 - 30.119$\\
			$R = 3591 - 3570$\\
			$R = 21 $ 



	\item 
		\begin{tabularx}{\linewidth}{ p{14.5cm} p{3cm} }
		Sedlack's reduction -\textit{1987}- & $0\%$ 
		\end{tabularx}
		
	\item 
		\begin{tabularx}{\linewidth}{ p{14.5cm} p{3cm} }
		Quisquater reduction -\textit{1990}- & $0\%$  
		\end{tabularx}	
		Source: \cite{birthday-2012-joye}		  
		Patent protected
	\item 
		\begin{tabularx}{\linewidth}{ p{14.5cm} p{3cm}}
		ZDN algorithm -\textit{Infineon technologies 1990's}-  & $0\%$  
		\end{tabularx}
			
		\noindent	
		A far effective, hardware-compatible algorithm, ZDN based modular multiplication
		was developed by Infineon. ZDN based modular multiplication replaces the
		multiplication and reduction operations with a single operation, which the
		system can execute in a single clock-cycle. This algorithm implements a 
		look-ahead Booth (LABooth) multiplication with ZDN based 
		(Zwei Drittel N , 2/3N in German) modular reduction (LARed). 
		This algorithm further improves on register constraints because it ensures
		that the partial product remains at approximately 2/3N. With this algorithm
		the multiplication and modular reduction are calculated completely in parallel
		WTF!.
		\vspace*{3mm}

		VS side channel cryptanalysis:\\
		* immune against SvsM discrimination		  
\end{itemize} 

\textit{Barrett vs Montgomery}\\
Similarities:\\
    Both require pre-computing various constants for a given modulus $N$. Their input range is $[0,N^2)$. \\
    Their last las step lies in $[0,2n)$, with a final fix-up step to reach the output range of $[0,n)$.\\
    They perform 2 internal multiplications per reduction.\\
    Their reduction phases avoid division or remainder by non-powers-of-2.

Differences:\\
    Montgomery reduction requires expensive conversion into and out of 'Montgomery form', whereas Barrett reduction operates on regular numbers directly. \\
    Montgomery is based on modular congruences and exact division, whereas Barrett is based on approximating the real reciprocal with bounded precision. \\

Consumption: ''\\
    If the modulus is $n$ bits long, 
    Montgomery: two $n$-by-$n$ bit multiplications yielding $2n$ bits. 
    Barrett: one $2n$-by-$n$ bit multiplication yielding $3n$ bits, plus a $n$-by-$n$ bit multiplications yielding $2n$ bits, 

Usage:\\
    Montgomery reduction is suitable for modular exponentiation but not for working with various unrelated numbers where Barrett reduction is a good candidate for both,but more expensive.\\