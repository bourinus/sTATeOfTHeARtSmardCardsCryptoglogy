\end{itemize}
\subsubsection{Optimized pre-computations}
\begin{itemize}
	\item  	
		\begin{tabularx}{\linewidth}{ p{16cm} p{1.5cm}}
		LtoR Optimized $2^k$-ary exponentiation: -\textit{?}-  & $0\%$
		\end{tabularx}	
			\noindent
			Optimization of the previous method, only even power of $x$ are stored only. 
			Require $ 2^{k-1}-1 $ precomputed values overall.
			\vspace{3mm}
			\begin{algorithm}[h]
				\KwIn{$x$, ${d =  d_{t-1} d_1 d_0}_{2^k}$}
				$y \leftarrow 1$	\;	

				$x_2 \leftarrow x^2$ \;	
				$\forall i = 0    {k-1}/2$, $x_{2i+1} = x^{2i-1} \times x_2$ \;		 
				\For{$ i = 0 \to t-1$ }
				{	
					 $y = x^{2^k}$ \;
					 \textbf{define:} $d_i = 2^{h_i} \times u_i $ 
					 where $u_i$ is odd and $ h_i $ maximal	\;		
					$y = {( y^{2^{k-h_i}} \times x_{u_i} )^2}^{h_i} $
				}										 
				\Return{$ y $}\;
				\caption{Optimized  LtoR $2^k$-ary method}
			\end{algorithm}			
						
		\underline{Example:} LtoR Optimized $2^k$-ary exponentiation\\
			Let's assume that $m=8$ \textit{i.e.} $k=3$ with $n=326_{10}=101\;000\;110_2$ 
			the method start scanning the bits from their MSB but three by three.
			The precomputed values are:   $x^3$, $x^5$, $x^7$.
			\begin{tabularx}{\linewidth}{ p{2cm} p{2cm} p{2cm} p{8cm} p{2cm}}
				$y:= 1$ & & & &\\
				$y := y ^2$ & $y := y ^2$ & $y := y ^2$ 			& 
				$y := y  \times  x^5$   &	$(y=x^5)$ \\
				$y := y ^2$ & $y := y ^2$ & $y := y ^2$ 			&
				$y := y  \times  x^0$ & $(y=x^{40})$\\
				$y := y ^2$ & $y := y ^2$ & $y := y  \times  x^3 $	& 
				$y := y  \times  x^2$ & $(y=x^{326})$
			\end{tabularx}
			
			Practical remark:\\
			\begin{center}
			On a SPA point of view it is crucial to know whether or not\\
			the implementation is saving pre-computation !
			\end{center}						
			\begin{itemize}
				\item if precomputed value are optimized:
				the goal become to discriminate elevation to the $8^{th}$ power 
				from multiplication by $(x^i)_{ 2 \leq i \leq7 }$ and in this case 
				we will always have sequence of elevation to the $8^{th}$ power then 
				a multiplication (or not). But the pattern of the curve should match 
				exactly the key.
				
				\item in this case we will not always have sequence of elevation 
				to the $8^{th}$ power then a multiplication (or not). Compare the last
				 line of the previous example and the same line with the 8-arry method.
			\end{itemize}
			
		 \textit{Variants:}\\
			 * RtoL\\
			 * Atomic version\\
			 * Multiply always\\