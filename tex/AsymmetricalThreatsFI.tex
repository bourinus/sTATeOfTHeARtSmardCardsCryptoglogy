\newpage
\subsubsection{Physical threat: fault injection}

THE reference: 
\cite{book-2012-joye}, published in 2012, countains more than 400 bibliographical references


Due to the mathematical complexity of some laser attack, 
which a complete description would force me to do a lot of 'recall' 
about abstract maths and would dramatically increase the size of 
this document, the goal for this part are:
\begin{center}
1 - list the known papers, sum-up practical content.\\
2 - evaluate their efficiency in term of bit(s) per fault.\\
3 - evaluate their feasibility in term of countermeasures.\\
\end{center}

Two remarks, as stupids than fundamental:
\begin{center}
	All attacks by fault injection, assume a \textit{fault model}.\\
	\textbf{
	The more unprecise the model is the more realistic the attack.\\
	Many of the fault model are 'impossible' to check.
}
\end{center}

\subsection*{FI targetting straightforward RSA:}
\begin{itemize}
\item Boneh - \nocite{eurocrypt-1997} \lokiquote{eurocrypt-1997-boneh}\\
\textit{aka} paper countaining 'The Bellcore attack' , 'The Flip bit attack I'

\underline{Arround this paper:}
Sorry but there is no Dr Bellcore, 'Bellcore' stands for 
Bell Communications Research a formerly famous center of research 
now closed. The Bellcore attack is in reality the 
'Boneh-DeMillo-Lipton attack' among the first one to deals with FI.

\underline{\textbf{Paper's sum-up:}}\\
Present FI attack breaking RSA in its CRT version with one faulty
signature and no particular no fault model.
Then, authors consider attack Fiat-Shammir scheme, Schnorr's scheme, 
RSA in RtoL version assuming a fault in some register , breaking those system 
with a much more larger number of fault.

\textbf{Summing up \textit{CRT-RSA's vulnerability to hardware faults}} 
\textit{'The Bellcore attack'}

\underline{Type of attack:}\\
unknown $\&$ reproducible input attack - Dr Bellecore's version\\
known $\&$ not reproducible input attack - Dr Lenstra's version

\underline{Targets:}
Every RSA algorithm using the Chinese theorem of the remainders.

\underline{Fault model}
No particular fault model is assumed, any random value for $S_{p}^{'}$
respecting the condition $S-S^{'}$ is not divisible by $p$ is suitable 
for this attack.
Following Gauss, the authors decompose the recombination the following way:
\begin{center}
$S = a \times S_p + b \times S_q $
\end{center}
Then, they assume one of the partial encryption were faulted:
\begin{center}
$S^{'} = a \times S_{p}^{'} + b \times S_{q} $
\end{center}
Taking in account that statistically $S-S^{'}$ is not divisible by $p$
\begin{center}
$q = gcd(S-S^{'}, n)$
\end{center}
Informed that an important paper was to be published but ignoring the details,
Arjen Lenkstra found a version of this attack requiring only one faulty signature
and the message. Also exit another version of this attck by Jorn-Marc Schmidt,
in his master repport, finding the key with two faulty signastures and their messages.


\textbf{Summing up \textit{ Breaking other implementations of RSA}} 
\textit{'The Flip bit attack I'}

\underline{Type of attack:}
randomly choosen plaintext attack, correct signature not mandatory\\
reproductible plaintext are not necessary\\
adaptative recovering algorithm: fault from other model are tolerated.

\underline{Targetted:}
Targets straightforward RSA using the S\&M algorithm in LtoR version.
		
\underline{Fault model}
A single random bit has been flipped in the output register of the
S\&M algorithm in RtoL version.
 
\underline{Theorem:} (Efficiency)
 With probability at least 1/2 , the secret exponent s can be extracted
from a device implementing the first exponentiation algorithm by collecting
(n/m)log(n) faults and $O(2^m n^3 )$ RSA encryptions for testting motives, 
for any $1 \leq m \leq n$. For small public exponent d this takes $O(2^m n^4 )$ time. 
For random d it takes $O(2^m n^5 )$ time.

Notations:	\\
\begin{tabularx}{\linewidth}{ p{6cm} p{7.5cm} l}
	Variable & Description & Status\\ 
	$ l = \frac{n}{m} log_2(n)$ & number of faults & known\\
	$(M_i)_{1 \leq  i \leq l}$ & Set of random of random messages & known\\ 
	$(E_i)_{1 \leq  i \leq l}$ & Set of corresponding signatures & unknown\\ 
	$(E^{'}_i)_{1 \leq  i \leq l}$ & Set of faulted signatures & known\\ 
	$(k_i)_{1 \leq  i \leq l}$ & index of the of fautled loop for $E^{'}_i$ & unknown\\
    $ s_n s_{n-1} ... s_{1}$ & bit of the secret exponnent & unknown\\
    $ s_n s_{n-1} ... s_{k_i} $ & bit already guessed  & known\\
    $ s_{k_i-1} s_{k_i-2} ... s_{k_{(i-1)}} $ & to guess bits  & unknown\\    
\end{tabularx}	

	\begin{algorithm}[h]
		\KwIn{$x,n \in \mathbb{N}$, $x \leq n$, ${d =  d_{t-1} d_1 d_0}_2$}
		$y \leftarrow 1$	\;	
		\For{  \textsf{all lenght  } $r=1,2,... $ }
		{			 
	 		\For{  \textsf{all r-bits candidates  } $u = u_{k_i-1}u_{k_i-2}...u_{k_{i}-r} $ }
			{	
			\textsf{form full candidate: }		 
			 $\omega = \sum \limits_{j=k_i}^{n}       s_j 2^j +
			           \sum \limits_{j=k_i-r}^{k_i-1} u_j 2^j $\; 
			\textsf{test full candidate: }		  
			 	$\exists \, ? e \in \{0, ... ,n\} / (E^{'}_j \pm 2^e M_j^\omega)^d = M_j \mod N$\;
			 \If{yes }{output: $u_{k_i-1}u_{k_i-2}...u_{k_{i}-r} $}	
 			 \If{no } {reject candidate}		 		
			}		
		}									 
		\caption{Boneh's flip bit attack recovering algorithm}
	\end{algorithm}
	
	Finally, the set of index of the faulted loop for $E^{'}_i$  is assumed to be 
sorted thanks the natural order, consequence with probability $p> 50 \%$, 
we have: $ k_{i+1}-k_{i}<m$. This section of the article finishes with a proof
that false positive\textit{i.e.} wrong candidate that passed the test, are rare.

	


\item Bao - \lokiquote{spw-1997-bao} \\
\textit{aka} apper countaining the 'flip bit attack'

\underline{Around the attack:}
The attack presented by Sciventure is in fact the second one much more 
realistic in its fault model. One of the very first paper on the subject.

\textbf{Paper's sum-up:}
Attack the RSA algorithm in its straightforward version, the ElGamal signature
scheme, the Schnorr signature scheme, and the DSA. 
RSA is attacked in two different ways: the first attack aim to flip on bit of the
message, the other one aiming to flip on bit of the exponent.

\underline{Fault model:}
Unrealistic: the first fault model - precisely flip one bit  in $m^{2^i}$ -
appears to be completely unaplicable. Certainly that to flip one bit of an 
exponent is an (difficult but) achievable objective ...

\underline{Type of attack:}\\
randomly chosen $\&$ reproducible plain-text attack\\

\underline{Target:} S \& M algorithm in LtoR and RtoL version.

\underline{Counter measure}
Nothing said bout that, because the authors give 
their own counter measure to their attack.\\

\textbf{Summing up \textit{ Attacking the RSA Scheme}}
 \textit{'flip bit attack II'}

\textbf{Notation:} let $m$ be the plain-text, $c$ the cyphertext and $t$ and
their number of bits, with this, the authors define, which allow them to 
write the cypher text as a product:
\begin{center}
$\forall i \in [0,t-1]$ $m_i = {m^2}^i \mod n$\\
$c ={c_{t-1}}^{d_{t-1}}...{c_i}^{d_i}...{c_1}^{d_1}{c_0}^{d_0} \mod n$
\end{center}
\textbf{Attack I:}
the message has been faulted, with a single bit flip:
\begin{center}
$c' ={c_{t-1}}^{d_{t-1}}...{c'_i}^{d_i}...{c_1}^{d_1}{c_0}^{d_0} \mod n$
\end{center}
Then 
$ \frac{c'}{c} = \frac{{c'_i}^{d_i}}{{c_i}^{d_i}} $ can be evaluated, 
on the other hand can be also calculated the $t^2$ possible values,
if a match is found then $i$ is known, $d_i=1$.\\
\underline{Limitation}\\
* only one $m_i$ contain one bit of error \\
* no propagation of error is tolerated, if $m$ is modified, all $m_i$ are\\

\textbf{Attack II:}
Approximately the same faulting only one bit of the exponent.

\underline{Limitation}\\
* only one $d_i$ contain one bit of error \\



\item Joye - \lokiquote{cciam-1997-joye}\\
\textit{aka} 'The Flip bit attack III'

\textbf{Paper's sum-up:}

This paper is extended the work of Boneh and Boa 'Flipped bit attack I \& II' the following way.
Fisrt is recalled the previous attacks, then they propose an extension to LUC
\footnote{LUC is a public-key cryptosystem developed by a group of researchers in Australia and New
Zealand. The cipher implements the analogs of ElGamal -LUCELG-, Diffie-Hellman -LUCDIF-, and RSA 
-LUCRSA- over Lucas sequences.} cryptosystem, KMOV cryptosystem and finally give minors improovment.
\footnote{KMOV is an elliptic curve based analogue to RSA}

\item Yen - \lokiquote{ieeetc-2000-joye}\\
\textit{aka} 'Safe error attack' - 


\underline{\textbf{Paper's sum-up:}}\\
Authors introduce safe error, the attacker did modify something 
but it did not have any effect, from this information can be deduced. 
This attack is extremely generic and can be applied to a lot of situation. 
On the other there is no way to check that something has been changed.

\underline{Applicability:} none.

\underline{Remark}
this work has been continued by some author distinguishing
M safe error and C safe error.


\item Schmidt - \lokiquote{fdtc-2008-schmidt}

\textbf{Paper's sum-up:}\\
Present a FI attack breaking straightforward RSA skipping squarrings.
Applicable to most the exponentiation algorithm. 


\textbf{Summing up \textit{Attack}} LtoR S\&M

\underline{Fault model:}
be able to skipp a determined squaring.

\underline{Targetted:}
RtoL \& LtoR: S\&M, S\&MA, $2^k$-arry method, sliding window.

\underline{Type of attack:}
random known plaintext attack $\&$ reproducible input attack, 
correct signature mandatory. 

\underline{Counter measures}
Ineffective: Square \& Multiply always.
Effective: the authors are claiming that they can overcome 
'most of SPA countermeasure' by skipping the phase where this counter
measure is applied. With their super fault model this trivial:
-'each time that I want to skipp an operation it works'-. 
Practically, 'most of SPA countermeasure' defeats this attack.

Initialization:\\
get $Sig_0$ -skip the last squaring- and the correspondent non faulted signature.\\
Then, if the last square has been genuinely skipped, we shall have:
\begin{center}
$
Sig = \left\{
    \begin{array}{ll}
        {Sig_0}^2 \mod n & \textsf{ for } e_0=0 \\
        {Sig_0}^2 \times m^{-1}  \mod n & \textsf{ for } e_0=1 \\
    \end{array}
\right.
$
\end{center}
Do this operation till the previous equation has been verified and then $e_0$ deduced.

Induction:\\
Then when all the first $k-1$ bit of the exponent has been obtained, 
if the right square has been genuinely skipped, we shall have:
\begin{center}
$
Sig_k = \left\{
    \begin{array}{ll}
        {Sig_{k-1}} \mod n & \textsf{ for } e_{k-1}=0 \\
        {Sig_{k-1}} \times m^{2^{k-1}}  \mod n & \textsf{ for } e_{k-1}=1 \\
    \end{array}
\right.
$
\end{center}


\item Boreale - \lokiquote{fdtc-2006-boreale}

\textbf{Paper's sum-up:}\\
Present a FI attacks breaking straightforward RSA in RtoL version using the Jacobi Symbol, using
a practical fault model: that external perturbation, or glitch, may cause a single modular 
multiplication to produce a truly random result. Two attacks are presented, the second one having relaxed condition.

\underline{Type of attack:}
known $\&$ reproducible plaintext attack .

\underline{Targetted:} RtoL $S\&M$ only.

\underline{Counter measures:}\\
Unefficient: Blind masking of the message: $m$ replaced by $r^e \times m \mod N$.\\
Efficient: verify the signature by checking that $S^e = m \mod N$, exponnent masking,
various delays, modulus blinding-?-.

\underline{Fault model:}
Practical: change the result of a certain multiplication for a random one.

\underline{Mathematical background:}
Jacobi symbol generalizes Legendre's ones, which value 
$\left( \frac{a}{p} \right)$, for $p$ prime, means:
\begin{center}
$
\left( \frac{a}{p} \right) = \left\{
    \begin{array}{ll}
      \;\; 1 & \textsf{if} \;\;\exists \; x \neq 0 \in \mathbb{Z}/{p\mathbb{Z}}  / a = x^2 \mod p\\
      -1     & \textsf{if} \not{\exists} \; x \neq 0 \in \mathbb{Z}/{p\mathbb{Z}}  / a = x^2 \mod p\\
       \;\;0 & \textsf{if} \;\;a = 0 \mod p\\
    \end{array}
\right.
$
\end{center}


\href{http://en.wikipedia.org/wiki/Jacobi_symbol}{ A link: Wiki about Jacobi \& legendre symbols: }

\textbf{Summing up \textit{Attack}} LtoR $S\&M$
		
Assumptions:\\
1- Each modular multiplication/squaring operation takes a constant time, 
say $\delta$ clock cycles, and $\delta$ is a constant known to the attacker.\\
2- Time taken by control-flow instructions is ignored, we view the
algorithm as a sequence of modular multiplications. 
Each phase $i$ takes either $\delta$ or $2\delta$.\\
3- A glitch applied onto the device during the execution of a modular multiplication
will result in a random value $r\in \mathbb{Z}/{2^s\mathbb{Z}}$ to be written in 
the involved register in place of the multiplication's correct result.\\
4- For message m it is assumed that $\left( \frac{m}{N} \right)=1$, 
if equal to $-1$ some equation shall be slighlty modified, the case where 
$\left( \frac{m}{N} \right)=0$ in unlikely : it implies $m=p  \textsf{ or } q$

The authors begin to define $T_i$ the moment when happend the $i^{th}$ operation 
while an encryption is performed. As the attacker already knows $d_0$ 
the first fault is done around  $t=T_1$, more precisely, $T_1 > t > T_1 -\delta$,
this will provok a fault in the squarring of the phase $i-1$.

The obtained signature can be written the following way, using the classical notation 
$c_i = m^{2^i} \mod n$
\begin{center}
$
S^{'} = 
c_0^{d_0}   c_1^{d_1}     \; \hdots \;         c_{i-1}^{d_{i-1}}
(r)^{d_i} (r^2)^{d_{i+1}} \; \hdots \; (r^{2^{l-i-1}})^{d_{l-1}}
 \mod n
$
\end{center}
Taking in account hypothesis 4, then it is clear that, except $ (r)^{d_i} $, 
each divisor of $S^{'}$ has Jacobi Symbol different from -1.
Therefore $\left( \frac{S^{'}}{N} \right)= -1$ implies
$\left( \frac{r^{d_i}}{N} \right)= -1$ and then $d_i = 1$. 
On the other hand to obtain $\left( \frac{S^{'}}{N} \right) \neq -1$ 
suggest that the more probable is that $d_i =0$. 

Authors finishes this part with an evaluation of the
probability $\{ d_i=1 \| \left( \frac{S^{'}}{N} \right) \neq -1 \}$.

If the moment of the $i^{th}$ operation is difficult to estimate,
the attack is run severasl time $ \approx 50$.

\underline{Software simulation:}\\
On a 768-RSA, 5000 faults are enough to recover, in 30 minutes,
the whole key in 70\% of the cases.





	
	

\end{itemize}

\subsection*{FI targetting CRT-optimized RSA:}
\begin{itemize}
\item Coron - \cite{ches-2009-coron}

\underline{Type of attack:}\\
reproducible input\\
partially known plain-text \\
RtoL exponentiation algorithm only\\
\underline{Result:}\\
CRT: one faulty encryption is enough\\
non CRT: several faulty encryption are required.\\
\underline{Fault localisation}:\\
CRT:\\
non CRT: one of the two CRT exponentiation\\
\underline{Relying on}:\\
Recent improvement of Coppersmith's algorithm to find small 
roots of multivariate polynomial.


\item Amhuller - Fault attack on CRT-RSA concrete 
result and practical approach - 2007
\item Coron - Fault Attacks and Countermeasures on 
Vigilants RSA-CRT Algorithm - 2010
\item Naccache - Modulus fault attack aginst RSA-CRT Signatures - CHES2011
\item Fouque - Attacking RSA-CRT signatures with fault 
Montgommery Multiplication - CHES2012
\end{itemize}
Electro magnetic fault injection are the most powerful... wtf
