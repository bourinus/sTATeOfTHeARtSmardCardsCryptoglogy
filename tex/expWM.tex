\newpage
\subsubsection{Scanning digits in base $2^k$}
\begin{itemize}
\item  	
		\begin{tabularx}{\linewidth}{ p{16cm} p{1.5cm}}
		Window Method: the LtoR $2^k$ary method: -
		\textit{Brauer 1939}-  & $0\%$  
		\end{tabularx}	
			\noindent
				Generalisation of the previous method, the bits of the exponent, in base 2, 
				are scanned $k$ by $k$, this method requires pre-computed values.
				Principle: $k$ squares are done each time scanning $k$ other bits of the exponent
				A multiplication by one of the $2^{k}-3$ pre-computed value is done 
				depending on the scanned group of bits..

				Only the precompuation of the $x^{d_j}$ such that $d_j$ appears in the representation of $d$ is needed.
				
			\begin{figure}[h]
			\begin{center}
	       		\includegraphics[scale=0.33]{images/2k.png}
				\caption{LtoR $2^k$ary method illustration, with $k=3$}
			\end{center}
			\end{figure}
			\begin{algorithm}[h]
				\KwIn{$x,n \in \mathbb{N}$, $x \leq n$, ${d =  d_{t-1} d_1 d_0}_2$}
				$x_0 \leftarrow 1$	\; 				
				\For{$ i = 0 \to 2^{k}-1$ }
				{						     
					 $x_i \leftarrow x_{i-1} \times x$ \;
				}
				$y \leftarrow 1$	\;
				\For{$ i = t-1 \to 0$ }
				{			 
					 $y = x^{2^k}$ \;
					 \If{$d_i > 0$}{
					 $y = y \times x_i$\;
					 } 	
				}									 
				\Return{$ y $}\;
				\caption{LtoR $2^k$-ary method}
			\end{algorithm}			

			\underline{Example 1:} LtoR $2^k$-ary method\\
			Let's assume that $m=8$ \textit{i.e.} $k=3$ with $n=326_{10}=101\;000\;110_2$ 
			the method start scanning the bits from their MSB but three by three.
			The precomputed values are:   $x^2$, ..., $x^7$\\
			\begin{tabularx}{\linewidth}{ p{2cm} p{2cm} p{2cm} p{8cm} p{2cm}}
				$y:= 1$ & & & &\\
				$y := y ^2$ & $y := y ^2$ & $y := y ^2$ & $y := y  \times  x^5$ & $(y=x^5)$ \\
				$y := y ^2$ & $y := y ^2$ & $y := y ^2$ & $y := y  \times  x^0$ & $(y=x^{40})$\\
				$y := y ^2$ & $y := y ^2$ & $y := y ^2$ & $y := y  \times  x^6$ & $(y=x^{326})$
			\end{tabularx}
			\vspace{3mm}
			
		\underline{Example 2:} LtoR $2^k$-ary method \\	
			Let's assume that:
			$n=11651101_{10}=101\;\;100\;\;011\;\;100\;\;100\;\;000\;\;011\;\;101_{2}$\\
			\begin{tabularx}{\linewidth}{ p{2cm} p{2cm} p{2cm}  p{8cm} p{4cm}}
				$y:= 1$	 &   		&          & $y:=y \times x^5$ & $(y=x^{5})$ \\
				$y:=y^2$ & $y:=y^2$ & $y:=y^2$ & $y:=y \times x^4$ & $(y=x^{44})$\\
				$y:=y^2$ & $y:=y^2$ & $y:=y^2$ & $y:=y \times x^3$ & $(y=x^{355})$ \\               
				$y:=y^2$ & $y:=y^2$ & $y:=y^2$ & $y:=y \times x^4$ & $(y=x^{2\,844})$\\
				$y:=y^2$ & $y:=y^2$ & $y:=y^2$ & $y:=y \times x^4$ & $(y=x^{22\,756})$\\       
				$y:=y^2$ & $y:=y^2$ & $y:=y^2$ & 				   & $(y=x^{182\,048})$\\
				$y:=y^2$ & $y:=y^2$ & $y:=y^2$ & $y:=y \times x^3$ & $(y=x^{1\,456\,387})$\\                 
				$y:=y^2$ & $y:=y^2$ & $y:=y^2$ & $y:=y \times x^5$ & $(y=x^{11\,651\,101})$\\		
			\end{tabularx}
			To get this result a certain decomposition of $n$ 
			was used which lead to 21 squarings and 7 multiplications.
			
			\underline{Name \& convention}							
				The $2^k$ary method may also be called $k$-bits Window method, 
				insisting on the number of bit scanned at each recursion...
			\newpage
	\item  	
		\begin{tabularx}{\linewidth}{ p{16cm} p{1.5cm}}
		Window Method: LtoR Optimized $2^k$-ary exponentiation: -  & $0\%$
		\end{tabularx}	
			\noindent
			Optimization of the previous method, only even power of $x$ are stored only.\\
			Require $ 2^{k-1}-1 $ precomputed values overall.
			\vspace{3mm}
			\begin{algorithm}[h]
				\KwIn{$x$, ${d =  d_{t-1} d_1 d_0}_{2^k}$}
				$x_0 \leftarrow 1$	\; 
				$x_1 \leftarrow x$	 \; 
				$x_2 \leftarrow x^2$  \; 				
				\For{$ i = 1 \to 2^{k-1}-1$ }
				{						     
					 $x_{2i+1} \leftarrow x_{2i-1} \times x_2$ \;
				}
				$y \leftarrow 1$ \;
				\For{$ i = t-1 \to 0$ }
				{			 
					 $y = x^{2^k}$ \;
					 \textbf{define:} $d_i = 2^{h_i} \times u_i $ 
					 where $u_i$ is odd and $ h_i $ maximal	\;		
					$y = {( y^{2^{k-h_i}} \times x_{u_i} )^2}^{h_i} $
				}									 
				\Return{$ y $}\;
				\caption{Optimized LtoR $2^k$-ary method}
			\end{algorithm}			
			
		\underline{Example:} LtoR Optimized $2^k$-ary exponentiation\\
			$n=326_{10}=101\;000\;110_2$ and $k=3$ \textit{i.e.} $m=8$ 
			the method starts scanning the bits from their MSB but three by three.
			The precomputed values are:   $x^3$, $x^5$, $x^7$.\\
			\begin{tabularx}{\linewidth}{ p{2cm} p{2cm} p{2cm} p{8cm} p{2cm}}
				$y:= 1$ & & & &\\
				$y := y ^2$ & $y := y ^2$ & $y := y ^2$ 			& 
				$y := y  \times  x^5$   &	$(y=x^5)$ \\
				$y := y ^2$ & $y := y ^2$ & $y := y ^2$ 			&
				$y := y  \times  x^0$ & $(y=x^{40})$\\
				$y := y ^2$ & $y := y ^2$ & $y := y  \times  x^3 $	& 
				$y := y  \times  x^2$ & $(y=x^{326})$
			\end{tabularx}
			
			Practical remark:
			\begin{center}
			On a SPA point of view it is crucial to know whether or not the 
			implementation is saving pre-computation !
			\end{center}						
			\begin{itemize}
				\item if precomputed value are optimized:
				the goal become to discriminate elevation to the $8^{th}$ power 
				from multiplication by $(x^i)_{ 2 \leq i \leq7 }$ and in this case 
				we will always have sequence of elevation to the $8^{th}$ power then 
				a multiplication (or not). But the pattern of the curve should match 
				exactly the key.
				
				\item in this case we will not always have sequence of elevation 
				to the $8^{th}$ power then a multiplication (or not). Compare the last
				 line of the previous example and the same line with the 8-arry method.
			\end{itemize}

\newpage
\item  	
		\begin{tabularx}{\linewidth}{ p{16cm} p{1.5cm}}
		Window Method: RtoL $2^k$ary method:   & $0\%$  
		\end{tabularx}	
			\noindent

			\begin{algorithm}[h]
				\KwIn{$x,n \in \mathbb{N}$, $x \leq n$, ${d =  d_{t-1} d_1 d_0}_2$}
				$x_0 \leftarrow 1$	\; 				
				\For{$ j = 1 \to 2^{k}-1$ }
				{						     
					 $R_j \leftarrow 1$ \;
				}
				$y \leftarrow x$	\;
				\While{$ m \leq n $}
				{			 
					 $d \leftarrow n \mod m $\;
					 \If{$d \neq 0$}{
					 $R[d] \leftarrow R[d]  \times y$\;
					 } 	
					 $y \leftarrow y^m$\;
					 $n \leftarrow \lfloor n/m \rfloor $\;			
				}
				$R[n] \leftarrow R[n] \times y$\;
				$ y \leftarrow R[m-1]$\;
				\For{$ j = m-2 \to 1$ }
				{						     
					$ R[j] \leftarrow R[j] \times R[j+1]$\;
				    $ y \leftarrow y \times R[j]$\;
				}					 
				\Return{$ y $}\;
				\caption{RtoL $2^k$-ary method}
			\end{algorithm}	

			\underline{Example :} RtoL $2^k$-ary method, $m=8$ \textit{i.e.} $k=3$ with $n=326_{10}$
			
			Initialization phase:\\
			$R = \{1,1,1,1,1,1,1\}$, $y:= x$, $m=8$, $n=326$

			While loop:\\
			$d = 326 \mod 8 = 6$, $R = \{1,1,1,1,1,x,1\}$, $y:= x^8$,  $n=40$\\
			$d =  40 \mod 8 = 0$, $R = \{1,1,1,1,1,x,1\}$, $y:= x^{64}$, $n= 5$

			End while:\\
			$R = \{1,1,1,1,x^{64},x,1,1\}$, $y:= 1$

			Recomposition phase:\\
			$j=6, R = \{1,		1,		1,		1,			x^{64},	x,	1\}$, $y:= x$\\
			$j=5, R = \{1,		1,		1,		1,			x^{65},	x,	1\}$, $y:= x^{66}$\\
			$j=4, R = \{1,		1,		1,		x^{65},		x^{65},	x,	1\}$, $y:= x^{131}$\\
			$j=3, R = \{1,		1,		x^{65},	x^{65},		x^{65},	x,	1\}$, $y:= x^{196}$\\
			$j=2, R = \{1,		x^{65},	x^{65},	x^{65},		x^{65},	x,	1\}$, $y:= x^{261}$\\
			$j=1, R = \{x^{65},	x^{65},	x^{65},	x^{65},		x^{65},	x,	1\}$, $y:= x^{326}$\\


		\textit{Variants:}\\			
			 right to left version, optimized version 
\end{itemize}		 