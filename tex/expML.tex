\newpage
\subsubsection{The Montgomery Ladder}
\begin{itemize}
	\item
		\begin{tabularx}{\linewidth}{ p{16cm} p{1.5cm} }
		Montgomery ladder technique - \lokiquote{ches-2002-joye} -  & $0\%$ 
		\end{tabularx}			
			\begin{algorithm}[h]
				\KwIn{$x$, ${d =  d_{t-1} d_1 d_0}_2$}
				\KwOut{ $ y   =  x^d \mod n$ }
				$ R_0 \leftarrow 1$ \; $ R_1 \leftarrow x$ \;		
				\For{$i  = t-1$ \textbf{to} $0$}{			 
					\If{ $d_i= 0$ }{ 	
						$ R_1 \leftarrow R_0 \times R_1$\;
						$R_0 \leftarrow {R_0}^2 $	}										
					\Else{ 						
						$ R_0 \leftarrow R_0 \times R_1$\;
						$ R_1 \leftarrow {R_1}^2 $	 
						} 													 	
				}									 
				\Return{$ R_0 $}
				\caption{Montgomery ladder technique}
			\end{algorithm}			
			
%  * R0 := 0
%  * R1 := P
%  * for i from m to 0 do
%     * if di = 0 then
%        * R1 := R0 + R1
%        * R0 := 2R0
%     * else
%        * R0 := R0 + R1
%        * R1 := 2R1
%  * Return R0
			
		\underline{Example:} Montgomery ladder technique: \\
			$n=23_{10}=10111_2$
			
			\begin{tabularx}{\linewidth}{ p{2cm} p{2cm} }
				$R_0=1$ 		& $R_1=x$ \\
				$R_0=x^{1}$		& $R_1=x^{2}$ \\
				$R_0=x^{2}$ 	& $R_1=x^{3}$ \\
				$R_0=x^{5}$ 	& $R_1=x^{6}$ \\
				$R_0=x^{11}$ 	& $R_1=x^{12}$ \\
				$R_0=x^{23}$ 	& $R_1=x^{24}$ \\
			\end{tabularx}	
			Note that the following powers of $x$ are successively computed:
			$x$ $x^2$ $x^3$ $x^4$ $x^5$ $x^6$  $x^{11}$ $x^{12}$ $x^{23}$ $x^{24}$.
			
		\textit{VS side channel cryptanalysis}\\			
			This technique is vulnerable to the doubling attack
\end{itemize}